%%%%%%%%%%%%%%%%%%%%%%%%%%%%%%%%%%%%%%%%%
% Beamer Presentation
% LaTeX Template
% Version 1.0 (10/11/12)
% This template has been downloaded from:
% http://www.LaTeXTemplates.com
%
% License:
% CC BY-NC-SA 3.0 (http://creativecommons.org/licenses/by-nc-sa/3.0/)
%%%%%%%%%%%%%%%%%%%%%%%%%%%%%%%%%%%%%%%%%

\pdfminorversion=4

\newcounter{aufgabe}

\documentclass{beamer}

\mode<presentation> {
	\usepackage[utf8]{inputenc}
	\usetheme{Hannover}
	\usecolortheme{dolphin}
}

\usepackage{graphicx}
\usepackage{booktabs}
\usepackage[T1]{fontenc}
\usepackage{inconsolata}
\usepackage{lmodern}

\usepackage{color}

\definecolor{pblue}{rgb}{0.13,0.13,1}
\definecolor{pgreen}{rgb}{0,0.5,0}
\definecolor{pred}{rgb}{0.9,0,0}
\definecolor{pgrey}{rgb}{0.46,0.45,0.48}

\usepackage{listings}
\lstset{language=Java,
  showspaces=false,
  showtabs=false,
  breaklines=true,
  showstringspaces=false,
  breakatwhitespace=true,
  commentstyle=\color{pgreen},
  keywordstyle=\color{pblue},
  stringstyle=\color{pred},
  basicstyle=\ttfamily,
  moredelim=[il][\textcolor{pgrey}]{$$},
  moredelim=[is][\textcolor{pgrey}]{\%\%}{\%\%}
}

\lstset{literate=%
    {Ö}{{\"O}}1
    {Ä}{{\"A}}1
    {Ü}{{\"U}}1
    {ß}{{\ss}}1
    {ü}{{\"u}}1
    {ä}{{\"a}}1
    {ö}{{\"o}}1
    {~}{{\textasciitilde}}1
}

\title[Einführung Java]{Einf\"uhrung in die Programiersprache Java}

\author{FokusNeugier}
\institute[HSRM] 
{
Hochschule Rheinmain\\www.fokusneugier.de\\
\medskip
\textit{Lizenz: CC BY-NC-SA 3.0} 

}
\date{\today} 

% --- begin document  --- %

\begin{document}

\begin{frame}
	\titlepage 
\end{frame}

\begin{frame}
	\frametitle{Overview}
	\setcounter{tocdepth}{1}
	\tableofcontents
\end{frame}

\begin{frame}
	\frametitle{Diese Pr\"asentation}
	Diese Pr\"asentation (Quellcode und PDF) kann heruntergeladen werden\\
	\url{https://github.com/FokusNeugier/java-kurs}
\end{frame}

\section{Einführung} 

\subsection{Überblick} 

\begin{frame}
	\frametitle{Warum Java?}
	\begin{itemize}
		\item Einfach zu lernen
		\item Wenig Aufwand für GUIs
		\item Android Apps, Desktop Apps, Mindstorms etc
	\end{itemize}
\end{frame}

\begin{frame}[fragile]
	\frametitle{Fangen wir an! - So sieht Java aus:}
	\begin{lstlisting}[language=java]
class HelloWorld {
  public static void main(String[] args) {
    System.out.println("Hello World!");
  }
}
	\end{lstlisting}
\end{frame}

\subsection{Variablen} 

\begin{frame}
	\frametitle{Was sind Variablen?}
	\begin{itemize}
		\item Speicher für verschiedene Daten (Zahlen, Zeichen, Texte...)
		\item Ein Teil der Datentypen: int, long, double, char, String
		\item Werden beschrieben durch: typ name = wert; Bsp: \lstinline[language=java]{int i = 4;}
		\item Programme können mit den Namen der Variablen arbeiten
		\item Können mit Oparatoren verknüpft werden
		\item Werden auch in der Mathematik angewendet $$Beispiel: \;\;x^2 + 2 = 4 \;\;\;\;\; x * 2 = x + x $$
	\end{itemize}
\end{frame}

\subsection{Operatoren}

\begin{frame}[fragile]
	\frametitle{Operationen}
	\begin{lstlisting}[language=java]
int x = 1;
int y = 2;
int z = 3;

x = y + x; // Addition
x = y - x; // Subtraktion
x = y * x; // Multiplikation
x = y / x; // Division

// Zeigt an, dass der Rest der Zeile ein Kommentar ist.
	\end{lstlisting}
\end{frame}


\subsection{Strings} 

\begin{frame}
	\frametitle{Strings}
	\begin{itemize}
		\item Zeichenketten oder Strings genannt
		\item Werden in " eingerahmt: \lstinline[language=java]{String name = "max";}
		\item Können mit dem '+' / Plus Operator aneinander gehängt werden. Weitere Operatoren für Strings gibt es nicht.
	\end{itemize}
\end{frame}

\subsection{String Code Beispiele}

\begin{frame}[fragile]
\frametitle{String Code Beispiele}
\begin{lstlisting}[language=java]

// Einen neuen String erstellen
String name = "max"; 

// String an String 'addieren'
name = name + ":" + name; 

// String in der Konsole anzeigen
System.out.println(name);

// Ausgabe von diesem Beispiel:
max:max

\end{lstlisting}
\end{frame}
\stepcounter{aufgabe}
\subsection{\arabic{aufgabe}. Aufgabe}

\begin{frame}[fragile]
	\frametitle{\arabic{aufgabe}. Aufgabe}
	\begin{itemize}
		\item Von jedem euch bekannten Typen an Variablen mindestens eine erstellen
		\item Mindestens jeder euch bekannte Operator einmal sinnvoll benutzt (führt z.B. ein paar Rechnungen durch)
		\item Zu allem sinnvolle Ausgaben
		\item Schreibt euren Code in die "Main Methode" einer neuen Java Datei und nutzt Eclipse, um den Code zu testen
		\item Stellt Fragen!
	\end{itemize}
\end{frame}

\section{Bedingungen}
\begin{frame}[fragile]
	\frametitle{Was sind Bedingungen}
	\begin{itemize}
		\item Programmierung wird erst dann interessant, wenn man zwischen Fällen unterscheiden kann
		\item Verzweigungen im Code, Code wird unterteilt in die möglichen Fälle
		\item Beispiel: Abhängig vom Geschlecht wird die Anrede zu Herr / Frau
	\end{itemize}
\end{frame}


\begin{frame}[fragile]
\subsection{if-Anweisung}
\frametitle{Die if-Anweisung}
\begin{lstlisting}[language=java]
if (BEDINGUNG) {
	// Anweisung wenn die Bedingung wahr ist	
}
else {
	// !OPTIONAL! Anweisung wenn die Bedingung falsch
}
	\end{lstlisting}
\end{frame}

\subsection{Boolsche Ausdrücke}
\begin{frame}[fragile]
\frametitle{Was ist ein Boolean?}
		
	\begin{itemize}
		\item Ein Boolean ist ein Datentyp
		\item Kann nur die Werte "true" (wahr) oder "false" (falsch) enthalten
		\item Die Bedindung in if-Anweisungen ist ein Boolean
		\item Wenn der Boolean den Wert "true" enthält wird der Code zwischen \{\} ausgeführt
		\item Wenn der Boolean den Wert "false" enthält wird der Code des else Blocks ausgeführt
	\end{itemize}
\end{frame}

\begin{frame}[fragile]
	\frametitle{Übersicht Boolsche Ausdrücke}
	Vergleiche ergeben Booleans:
	\begin{itemize}
		\item == gleich
		\item != ungleich
		\item \textgreater= größer gleich
		\item \textless= kleiner gleich
		\item \textless\ kleiner
		\item \textgreater\ größer
	\end{itemize}
\end{frame}

\begin{frame}[fragile]
	\frametitle{Kleine Übung}
	Was für booleans ergeben diese Vergleiche?
	\begin{itemize}
		\item 2 == 3
		\item 2 != 3
		\item 2 >= 2
		\item 2 > 2
		\item 3 <= 1
		\item 1 < 3
	\end{itemize}
\end{frame}

\begin{frame}[fragile]
	\frametitle{Kleine Übung}
	\begin{lstlisting}[language=java]
int zahl = 42;

// Was wird in der Konsole erscheinen?
if (zahl < 50) {
	System.out.println("if");
}
else {
	System.out.println("else");
}
\end{lstlisting}
\end{frame}

\section{Schleifen} 

\begin{frame}
	\frametitle{Was ist eine Schleife?}
	\begin{itemize}
		\item Mit Schleifen kann man Code Abschnitte wiederholen
		\item Es gibt unterschiedliche Arten von Schleifen:
		\item for
		\item while
		\item do ... while
	\end{itemize}
\end{frame}

\begin{frame}[fragile]
	\subsection{For Schleife}
	\frametitle{Die For Schleife} 
	\begin{lstlisting}[language=java]

// for (STARTWERT; BEDINGUNG; SCHRITT)
for (int i = 0; i < 10; i++) {
  System.out.println("Hallo, " + i);
}

	\end{lstlisting}
\end{frame}

\begin{frame}[fragile]
\subsection{While Schleife}
\frametitle{Die While Schleife} 
\begin{lstlisting}[language=java]

int a = 0;
int b = 2;
// while (BEDINGUNG)
while (a != b) {
  System.out.println("Hallo, " + a);
  a = a + 1;
  /* oder kürzer: a++ */	
}

	\end{lstlisting}
\end{frame}


\begin{frame}[fragile]
\subsection{Do While Schleife}
\frametitle{Die do While Schleife} 
\begin{lstlisting}[language=java]

int a = 2;
int b = 2;
// do { ... } while (BEDINGUNG);
do {
  System.out.println("Hallo, " + a);
  a = a + 1;	
} while(a != b);

	\end{lstlisting}
\end{frame}
\stepcounter{aufgabe}
\begin{frame}[fragile]
\subsection{\arabic{aufgabe}. Aufgabe}
\frametitle{\arabic{aufgabe}. Aufgabe}
	
\begin{lstlisting}[language=java]
int höhe = 5;

// Überlegt euch Code, der folgendes Muster ausgibt:
x
xx
xxx
xxxx
// Es sollen so viele Zeilen ausgegeben werden, wie in der Variable "höhe" steht.

	\end{lstlisting}
\end{frame}
\stepcounter{aufgabe}
\begin{frame}[fragile]
\subsection{\arabic{aufgabe}. Aufgabe}
\frametitle{\arabic{aufgabe}. Aufgabe}

\begin{lstlisting}[language=java]
int höhe = 5;

// Überlegt euch Code, der folgendes Muster ausgibt:
x
xx
x
xx
x
// Es sollen so viele Zeilen ausgegeben werden, wie in der Variable "höhe" steht.
	\end{lstlisting}
\end{frame}
\stepcounter{aufgabe}
\begin{frame}[fragile]
\subsection{\arabic{aufgabe}. Aufgabe}
\frametitle{\arabic{aufgabe}. Aufgabe}
	
\begin{lstlisting}[language=java]

int höhe = 5;

// Überlegt euch Code, der folgendes Muster ausgibt:
   x  
  xxx
 xxxxx
xxxxxxx
 xxxxx
  xxx
   x
// Es sollen so viele Zeilen ausgegeben werden, wie in der Variable "höhe" steht.


	\end{lstlisting}
\end{frame}

\section{Fuktionen}
\begin{frame}[fragile]
	\frametitle{Was sind Funktionen}
	\begin{itemize}
		\item Wiederkehrende Aufgaben kapseln
		\item Logische Trennung von Funktionalit\"at
		\item Code wird \"ubersichtlicher
		\item Funktionen haben sogenannte Parameter / R\"uckgabewerte
	\end{itemize}
\end{frame}

\subsection{Aufrufe}
\begin{frame}[fragile]
\frametitle{Aufrufe}
	
	
\begin{lstlisting}[language=java]

// Funktion definieren
void sageHallo(){
	System.out.println("Hallo");	
}

// Das Wort "void" bedeutet, die Methode hat keinen Rückgabewert

// Funktion aufrufen
sageHallo();

	\end{lstlisting}
\end{frame}

\subsection{Parameter}
\begin{frame}[fragile]
	\frametitle{Fuktionen Parameter \"ubergeben}
	\begin{itemize}
		\item Wiederkehrende Aufgaben unterscheiden sich
		\item Aufruf mit Parameter: \lstinline[language=java]{meineFunktion(eineVariable);}
	\end{itemize}
\end{frame}

\begin{frame}[fragile]
	\frametitle{Ein Beispiel}
		
	\begin{lstlisting}[language=java]

// Funktion definieren
void sageHallo(String name){
	System.out.println("Hallo " + name);	
}

// Funktion aufrufen
sageHallo("max");

	\end{lstlisting}
\end{frame}

\subsection{R\"uckgabe}
\begin{frame}[fragile]
	\frametitle{Fuktionen mit R\"uckgabe}
	\begin{itemize}
		\item Eine Funktion kann ein Ergebnis zur\"uckliefern
		\item Aufruf mit R\"uckgabe: \lstinline[language=java]{int uhr = holeUhrzeit();}
	\end{itemize}
\end{frame}

\begin{frame}[fragile]
	\frametitle{Ein Beispiel}
	\begin{lstlisting}[language=java]

// Fuktion definieren
int addiere(int x, int y){
	return x + y;	
}

// Fuktion aufrufen
int i = addiere(1, 2);

System.out.println(i);

	\end{lstlisting}
\end{frame}
\stepcounter{aufgabe}
\subsection{\arabic{aufgabe}. Aufgabe}
\begin{frame}[fragile]
	\frametitle{\arabic{aufgabe}. Aufgabe}
		
	\begin{lstlisting}[language=java]
// Schreibe eine Funktion, die je nach Geschlecht "Hallo, Herr/ Frau Name" und den Namen ausgibt.
public void sageHallo(String name, boolean istFrau) {
	
}
	\end{lstlisting}
\end{frame}
\stepcounter{aufgabe}
\subsection{\arabic{aufgabe}. Aufgabe}
\begin{frame}[fragile]
	\frametitle{\arabic{aufgabe}. Aufgabe}
	\begin{lstlisting}[language=java]
// Schreibe eine Funktion die das Quadrat einer Zahl zurückgibt.
int pow(int x, int y) {
	return 0;
}
	\end{lstlisting}
\end{frame}
\end{document} 
